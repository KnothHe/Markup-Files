% Compile With XeLaTeX

\documentclass{beamer}
\usepackage{ctex}
\usepackage{amsmath}
\usepackage{amssymb}
\usepackage{amsfonts}
\usepackage{listings}

\usepackage{xcolor}

\definecolor{mygreen}{rgb}{0,0.6,0}
\definecolor{mygray}{rgb}{0.5,0.5,0.5}
\definecolor{mymauve}{rgb}{0.58,0,0.82}

\lstset{ 
    backgroundcolor=\color{white},   % choose the background color; you must add \usepackage{color} or \usepackage{xcolor}; should come as last argument
    basicstyle=\ttfamily\footnotesize,        % the size of the fonts that are used for the code
    breakatwhitespace=true,         % sets if automatic breaks should only happen at whitespace
    breaklines=true,                 % sets automatic line breaking
    captionpos=b,                    % sets the caption-position to bottom
    commentstyle=\color{mygreen},    % comment style
    deletekeywords={...},            % if you want to delete keywords from the given language
    escapeinside={\%*}{*)},          % if you want to add LaTeX within your code
    extendedchars=true,              % lets you use non-ASCII characters; for 8-bits encodings only, does not work with UTF-8
    frame=single,                   % adds a frame around the code
    keepspaces=false,                 % keeps spaces in text, useful for keeping indentation of code (possibly needs columns=flexible)
    morekeywords={bit, sbit, sfr, sfr16},            % if you want to add more keywords to the set
    keywordstyle=\color{blue},       % keyword style
    language=C,                 % the language of the code
    numbers=left,                    % where to put the line-numbers; possible values are (none, left, right)
    numbersep=5pt,                   % how far the line-numbers are from the code
    numberstyle=\tiny\color{mygray}, % the style that is used for the line-numbers
    rulecolor=\color{black},         % if not set, the frame-color may be changed on line-breaks within not-black text (e.g. comments (green here))
    showspaces=false,                % show spaces everywhere adding particular underscores; it overrides 'showstringspaces'
    showstringspaces=false,          % underline spaces within strings only
    showtabs=false,                  % show tabs within strings adding particular underscores
    %  stepnumber=2,                    % the step between two line-numbers. If it's 1, each line will be numbered
    stringstyle=\color{mymauve},     % string literal style
    tabsize=2,                 % sets default tabsize to 2 spaces
    title=\lstname                   % show the filename of files included with \lstinputlisting; also try caption instead of title
}

\usetheme{CambridgeUS}
%\usecolortheme{beaver}

\begin{document}

\title{C51 编程语言基础}
\author{贺广来}

\begin{frame}
    \titlepage
\end{frame}

\begin{frame}{Table of contents}
    \tableofcontents
\end{frame}


\section{C51 与标准 C 语言的比较}
\begin{frame}{C51 与标准 C 语言之间的差别}
    \begin{enumerate}
        \item
            库函数不同。不适合嵌入式控制器系统的库函数,被排除在 C51
            标准库中,例如字符屏幕和图形函数。有些函数则需要根据 8051
            的硬件特点进行调整。例如标准 c 语言的库函数中的 \texttt{prinf} 和
            \texttt{scanf} 通常用于屏幕打印和接受字符。
            \pause
        \item
            数据类型有区别。在标准 C 语言的数据类型上添加了四种扩展的数据类型。
            \pause
        \item
            C51 语言中的变量存储模式与标准 C 语言中的变量存储模式不同。
            \pause
        \item
            数据存储类型不同。
            \pause
        \item
            标准 C 语言没有处理单片机中断的定义。
            \pause
        \item
            程序结构的差异。嵌套层次不能过多,普通函数不支持递归。
    \end{enumerate}
\end{frame}

\section{C51 程序设计基础}

\subsection{C51 语言中的数据类型与存储类型}

\subsubsection{数据类型}

\begin{frame}{数据类型}
    \begin{enumerate}
        \item
            char (signed \& unsigned) \textbf{8b,1B} (可以将 char 类型看作是 1B 的
            int 类型)
            \pause

            \begin{itemize}
                \item
                    unsigned char 取值范围 0 \textasciitilde{} 255
                \item
                    signed char 取值范围 -128 \textasciitilde{} +127
                    (正负数的不对称性是由于补码编码的特性造成的)
            \end{itemize}
            \pause
        \item
            int (signed \& unsigned) \textbf{16b,2B}

            \pause
            \begin{itemize}
                \item
                    unsigned int 取值范围 0 \textasciitilde{} 65536
                \item
                    signed int 取值范围 -32768 \textasciitilde{} +32767
                    \pause
            \end{itemize}
        \item
            long (signed \& unsigned) \& float \& double \textbf{32b,4B}

            \begin{itemize}
                \item
                    此处 float 和 double 可看作是一种数据类型
            \end{itemize}
            \pause
        \item
            bit \textbf{1b}
        \item
            sfr \textbf{8b,1B}
        \item
            sfr16 \textbf{16b,2B}
        \item
            sbit \textbf{1b}
    \end{enumerate}

    \textbf{此处有表格: Keil C51 支持的数据类型}
\end{frame}

\begin{frame}[fragile]{C51 的扩展数据类型}
    \begin{enumerate}

        \item
            位变量 bit

            \begin{itemize}

                \item
                    取值只能为 \texttt{1(true)} 和 \texttt{0(false)}
            \end{itemize}
            \pause
        \item
            特殊功能寄存器 sfr

            \begin{itemize}

                \item
                    8051 的特殊功能寄存器分布在片内 RAM 80H \textasciitilde{} 0FFH
                    之间。
                \item
                    例 \texttt{sfr\ P1\ =\ 0x90}
            \end{itemize}
            \pause
        \item
            sfr16

            \begin{itemize}

                \item
                    16 位的特殊功能寄存器。
                \item
                    例 \texttt{sfr16\ DPTR\ =\ 0x82}
            \end{itemize}
            \pause
        \item
            特殊功能位 sbit

            \pause
            \begin{lstlisting}[language=C]
                sfr PSW = 0xd0
                sbit OV = PSW^2
            \end{lstlisting}

    \end{enumerate}

    注意区分 bit 和 sbit。

\end{frame}

\begin{frame}{注释}
    \begin{enumerate}
        \item
            单行 //
        \item
            多行 /* */
    \end{enumerate}
\end{frame}

\subsubsection{数据存储类型}

\begin{frame}{数据存储类型}
    8051 单片机有片内 RAM、片外 RAM 和 程序存储区。
\end{frame}

\begin{frame}{片内 RAM 的数据存储区}
    \begin{itemize}
        \item
            存储区为 DATA,存储类型为 data,存储空间为片内 RAM 的低 128
            字节。直接寻址。
        \item
            存储区为 BDATA,存储类型为 bdata,存储空间为片内 RAM 的 20H
            \textasciitilde{} 2FH 的位寻址区。
        \item
            存储区为 IDATA,存储类型为 idata,存储空间为片内 RAM 的 256
            字节。间接寻址。
    \end{itemize}
\end{frame}

\begin{frame}{片外 RAM 的数据存储区}
    \begin{itemize}
        \item
            存储区为 XDATA,存储类型为 xdata,存储空间为片外 RAM 的 64KB
            空间。使用 @DPTR 间接寻址。
        \item
            存储区为 PDATA,存储类型为 pdata,存储空间为片外 RAM 的 256B
            空间。使用 @Ri 间接寻址。
    \end{itemize}
\end{frame}

\begin{frame}{程序存储区}
    \begin{itemize}
        \item
            存储区为 CODE,存储类型为 code,存储空间为程序存储区,使用 DPTR
            寻址。
    \end{itemize}
\end{frame}

\begin{frame}[fragile]{DTAT 区}
    \begin{itemize}
        \item
            寻址的速度最快。最好把经常使用的变量存放在该区域。
        \item
            存储空间有限。
        \item
            DATA 区除包含程序变量外,还包含堆栈和寄存器组。
        \item
            DATA 区存储类型标识符为 data。
        \item
            存储空间为 片内 RAM 的低 128 字节,直接寻址存储区。
    \end{itemize}

    \begin{lstlisting}[language=C]
    unsigned char data system_status = 0;
    unsigned int data unit_id[8];
    char data inp_string[20];
    \end{lstlisting}

    \begin{itemize}
        \item
            由于使用工作寄存器组传参。至少失去 8 字节的空间,用于传参。
        \item
            由于堆栈空间较小,容易堆栈溢出。堆栈溢出后会直接复位。
    \end{itemize}
\end{frame}

\begin{frame}[fragile]{BDATA 区}
    \begin{itemize}
        \item
            DATA 中的位寻址区。在该区中声明的变量可直接进行位寻址。
        \item
            BDATA 区存储类型标识符为 bdata。
        \item
            对应的存储空间为片内 RAM 的 16 字节存储区(20H \textasciitilde{}
            2FH)中的 128 个位。
        \item
            C51 编译器不允许在 BDATA 区声明 float 和 double 型的变量。
    \end{itemize}

    \begin{lstlisting}[language=C]
    unsigned char bdata status_byte;
    unsigned int bdata status_word;
    sbit sata_flag = status_byte^4;
    if (status_word^15) {
        ...
    }
    stat_flag = 1;
    \end{lstlisting}
\end{frame}

\begin{frame}[fragile]{IDATA 区}
    \begin{itemize}
        \item
            IDATA 区使用寄存器作为指针来间接寻址,常用于存放使用较为频繁的变量。
        \item
            IDATA 区存储类型标识符为 idata。
        \item
            对应的存储空间为片内 RAM 的 256
            字节存储区,只能间接寻址,速度比直接寻址慢。
    \end{itemize}

    \begin{lstlisting}[language=C]
    unsigned char idata system_status = 0;
    unsigned int idata unit_id[8];
    char idata inp_string[16];
    float idata out_value;
    \end{lstlisting}
\end{frame}

\begin{frame}[fragile]{PDATA 区和 XDATA 区}
    \begin{itemize}
        \item
            PDATA 区和 XDATA 区用于片外 RAM。
        \item
            PDATA 可寻址 256 字节(\(2^8\))。
        \item
            PDATA 区存储类型标识符为 pdata。
        \item
            XDATA 最多可寻址 64KB(\(2^{16}\))。
        \item
            XDATA 区存储类型标识符为 xdata。
        \item
            对 PDATA 区寻址比对 XDATA 区寻址快。
        \item
            由于外部 RAM 和 外部 I/O 口是统一编址的,因此外部 RAM
            除包含数据存储器外,还包含外部 I/O 口的地址。
    \end{itemize}

    \begin{lstlisting}[language=C]
    unsigned char xdata system_status = 0;
    unsigned int pdata unit_id[8];
    char xdata inp_string[16];
    float pdata out_value;
    \end{lstlisting}
\end{frame}

\begin{frame}[fragile]{CODE 区}
    \begin{itemize}
        \item
            只可读不可写。
    \end{itemize}

    \begin{lstlisting}[language=C]
    unsigned char code a[] = {0x00,0x01,0x02,0x03,0x04,0x05,0x06,0x07,0x08};
    \end{lstlisting}
\end{frame}

\subsubsection{数据存储模式}

\begin{frame}{数据存储模式}
    \begin{enumerate}
        \item
            SMALL

            \begin{itemize}

                \item
                    默认存储区为 DATA
            \end{itemize}
        \item
            COMPACT

            \begin{itemize}

                \item
                    默认存储区为 PDATA
            \end{itemize}
        \item
            LARGE

            \begin{itemize}

                \item
                    默认存储区为 XDATA
            \end{itemize}
    \end{enumerate}

\end{frame}

\subsection{C51 特殊功能寄存器及位变量定义}

\subsubsection{特殊功能寄存器的 C51 定义}

\begin{frame}[fragile]{1. 使用关键字 \texttt{sfr} 定义}
    \begin{itemize}
        \item
            格式: sfr 特殊功能寄存器的名称 = 特殊功能寄存器的地址
    \end{itemize}

    \begin{lstlisting}[language=C]
    sfr IE = 0xA8;
    sfr TCON = 0x88;
    sfr SCON = 0x98;
    \end{lstlisting}

    \begin{itemize}

        \item
            16 位 \texttt{SFR},使用 \texttt{sfr16} 关键字
    \end{itemize}

    \begin{lstlisting}[language=C]
    sfr16 DPTR = 0x82;
    \end{lstlisting}
\end{frame}

\begin{frame}[fragile]{头文件}
    \begin{itemize}

        \item
            使用头文件 \texttt{reg51.h}(或 \texttt{reg52.h})。
    \end{itemize}

    \begin{lstlisting}[language=C]
    #include <reg51.h>
    void main(void)
    {
        TL0 = 0xf0;
        TH0 = 0x3f;
        TR0 = 1;
    }
    \end{lstlisting}

\end{frame}

\begin{frame}[fragile]{特殊功能寄存器中的位定义}
    \begin{itemize}

        \item
            格式: 位名 = 特殊功能寄存器\^{}位置;
    \end{itemize}

    \begin{lstlisting}[language=C]
    sfr PSW = 0xd0;
    sbit CY = PSW^7;
    sbit OV = PSW^2;
    \end{lstlisting}

    \begin{itemize}

        \item
            格式: 位名 = 字节地址\^{}位置;
    \end{itemize}

    \begin{lstlisting}[language=C]
    sbit CY = 0xd0^7;
    sbit OV = 0xd0^2;
    \end{lstlisting}

\end{frame}

\begin{frame}[fragile]{特殊功能寄存器中的位定义}
    \begin{itemize}

        \item
            格式: 位名 = 位地址;
    \end{itemize}

    \begin{lstlisting}[language=C]
    sbit CY = 0xd7;
    sbit OV = 0xd2;
    \end{lstlisting}
\end{frame}

\begin{frame}[fragile]{「例」 AT89S51 单片机内 P1 口的各寻址位的定义}
    \pause

    \begin{lstlisting}[language=C]
    sfr P1 = 0x90;
    sbit P1_7 = P1^7;
    sbit P1_6 = P1^6;
    sbit P1_5 = P1^5;
    sbit P1_4 = P1^4;
    sbit P1_3 = P1^3;
    sbit P1_2 = P1^2;
    sbit P1_1 = P1^1;
    sbit P1_0 = P1^0;
    \end{lstlisting}
\end{frame}

\subsubsection{位变量的 C51 定义}

\begin{frame}[fragile]{C51 的位变量定义}
    \begin{itemize}

        \item
            C51 位变量使用 \texttt{bit} 关键字定义。
        \item
            格式: bit bit\_name;
    \end{itemize}

    \begin{lstlisting}[language=C]
    bit ov_flag;
    bit lock_pointer;
    \end{lstlisting}
\end{frame}

\begin{frame}[fragile]{C51 的函数可使用 \texttt{bit} 类型变量作为参数,也可将 \texttt{bit} 类型变量作为返回值}
    \begin{lstlisting}[language=C]
    bit func(bit b0,bit b1)
    {
        return (b1);
    }
    \end{lstlisting}
\end{frame}

\begin{frame}{位变量定义的限制}
    \begin{itemize}

        \item
            位变量定义不能用来定义指针和数组。
        \item
            在定义位变量时,允许定义存储类型,位变量均放入一个位段,此段总是位于片内
            RAM 中。因此其存储区限制为 DATA 和 IDATA。其他类型均不支持。
    \end{itemize}
\end{frame}

\subsection{C51 的绝对地址访问}

\begin{frame}[fragile]{绝对宏}
    \begin{itemize}

        \item
            引入头文件 \texttt{absacc.h}
        \item
            CBYTE 以字节形式对 code 区寻址。
        \item
            CWORD 以字形式对 code 区寻址。
            \pause
        \item
            DBYTE 以字节形式对 data 区寻址。
        \item
            DWORD 以字形式对 data 区寻址。
        \item
            XBYTE 以字节形式对 xdata 区寻址。
        \item
            XWORD 以字形式对 xdata 区寻址。
        \item
            PBYTE 以字节形式对 pdata 区寻址。
        \item
            PWORD 以字形式对 pdata 区寻址。
    \end{itemize}

\end{frame}

\begin{frame}[fragile]{绝对宏}
    \begin{lstlisting}[language=C]
    #include <absacc.h>
    #define PORTA XBYTE[0xffc0]
    #define NRAM DBYTE[0x50]
    \end{lstlisting}
\end{frame}

\begin{frame}[fragile]{「例」 片内 RAM 地址 \texttt{0x40}、片外 RAM 地址 \texttt{0xffc0}}
    \pause

    \begin{lstlisting}[language=C]
    #include <absacc.h>
    #define PORTA XBYTE[0xffc0]
    #define NRAM DBYTE[0x40]
    void main (void) 
    {
        PORTA = 0x3d;
        NRAM = 0x01;
    }
    \end{lstlisting}
\end{frame}

\begin{frame}[fragile]{\texttt{\_at\_} 关键字}
    \begin{itemize}

        \item
            可对指定的存储器空间的绝对地址进行访问。
        \item
            格式: {[}存储器类型{]} 数据类型说明符 变量名 \texttt{\_at\_}
            地址常数。
    \end{itemize}

    \pause
    \begin{lstlisting}[language=C]
    void main(void)
    {
        data unsigned char y1 _at_ 0x50;
        xdata unsigned int y2 _at_ 0x4000;
        y1 = 0xff;
        y2 = 0x1234;
        ...
        while(1);
    }
    \end{lstlisting}
\end{frame}

\begin{frame}[fragile]{「例」 将片外 RAM 的 2000H 开始的连续 20 字节单元清零}
    \pause
    \begin{lstlisting}[language=C]
    xdata unsigned char buffer[20] _at_ 0x2000;
    void main(void)
    {
        unsigned char i;
        for (i = 0; i < 20; i++) {
            buffer[i] = 0;
        }
    }
    \end{lstlisting}
\end{frame}

\begin{frame}[fragile]{「例」 将片内 RAM 的 40H 开始的连续 8 字节单元清零}
    \pause
    \begin{lstlisting}[language=C]
    data unsigned char buffer[8] _at_ 0x40;
    void main(void)
    {
        unsigned char i;
        for (i = 0; i < 8; i++) {
            buffer[i] = 0;
        }
    }
    \end{lstlisting}
\end{frame}

\subsection{C51 的基本运算}

\begin{frame}{C51 的基本运算}
    \begin{enumerate}
        \item
            算术运算符
        \item
            逻辑运算符
        \item
            关系元算符
        \item
            位运算
        \item
            指针和取地址运算符
    \end{enumerate}
\end{frame}

\begin{frame}[fragile]{「例」 编写程序将扩展某 I/O 口 PORTA(只能字节操作)的 PORTA.5 清零,PORTA.1 置一。}
    \pause
    \begin{lstlisting}[language=C]
    #include <absacc.h>
    #define PORTA XBYTE[0xffc0]
    void main(void)
    {
        ...
        PORTA = (PORTA & 0xdf) | 0x02;
        // PORTA = (PORTA & ~(1<<5)) | (1<<2);
        ...
    }
    \end{lstlisting}
\end{frame}

\begin{frame}[fragile]{指针和取地址运算符}
    \begin{lstlisting}[language=C]
    int a;
    int *b = &a; // 定义指向 int 型变量的指针变量 b 指向 int 型变量 a。在语句中,* 用于申明该变量为指针型变量,& 用于取地址。
    int c = *b; // 定义 int 型变量 c 的初始化值为指针变量 b 所指向的变量 a 的值。在此语句中,* 用于解引用。注意和上一句中的 * 的区别。
    \end{lstlisting}
\end{frame}

\subsection{C51 的分支与循环程序结构}

\begin{frame}
\begin{enumerate}

\item
  分支

  \begin{itemize}
  
  \item
    \texttt{if}
  \item
    \texttt{switch}
  \end{itemize}
\item
  循环

  \begin{itemize}
  
  \item
    \texttt{while}
  \item
    \texttt{do-while}
  \item
    \texttt{for}
  \end{itemize}
\end{enumerate}
\end{frame}

\subsection{C51 的数组}

\begin{frame}{C51 的数组}
\end{frame}

\subsection{C51 的指针}

\begin{frame}[fragile]{通用指针}
  \begin{itemize}
  
  \item
    共占 3
    个字节的存储空间。其中第一个字节为存储区类型,第二个和第三个字节分别是指针所指向的数据地址的高字节和低字节。
  \item
    优点是方便,缺点是访问速度慢。在所指向的目标变量存储类型不确定的时候使用。
  \end{itemize}

    \begin{lstlisting}[language=C]
    unsigned char *pz;
    \end{lstlisting}
\end{frame}

\begin{frame}[fragile]{存储器指针}
  \begin{itemize}
  
  \item
    在定义该指针变量时指明了存储器类型的指针。
  \end{itemize}

    \begin{lstlisting}[language=C]
    char xdata *str; // 指针变量 str 指向存储在 XDATA 区的 char 类型的变量
    int xdata *pd; // 指针变量 pd 指向存储在 XDATA 区的 int 类型的变量
    \end{lstlisting}

  \begin{itemize}
  
  \item
    由于指明了存储器的类型,相对于通用指针而言,第一个字节省略。
  \item
    DATA、BDATA、IDATA、PDATA 存储区,寻址空间在 256B 以内,指针仅占用 1
    字节。
  \item
    XDATA、CODE 寻址空间最大为 64KB,指针占用 2 字节。
  \end{itemize}

\end{frame}

\section{C51 的函数}

\begin{frame}[fragile]{C51 中的中断服务函数定义的格式}
    函数返回类型 函数名(行参列表) interrupt n [using n]

    \begin{itemize}
        \item 关键字 interrupt 后的 n 是中断号,取值为 0~4。
        \item 关键字 using 后的 n 是所使用的工作寄存器组,取值值为 0~3
        \item using 选项可省略。若未指明所使用的工作寄存器组,中断函数中的所有工作寄存器的内容将被保存堆栈中。
    \end{itemize}

    \begin{lstlisting}
    void int0() interrupt 0 using 1 /* 外部中断 0 的中断服务函数 */
    {
        /* 此处为中断服务函数的内容 */
    }
    \end{lstlisting}

\end{frame}

\end{document}
